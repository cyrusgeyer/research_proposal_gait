\documentclass[12pt,a4paper]{article}

% Packages
\usepackage[utf8]{inputenc}
\usepackage[margin=1in]{geometry}
\usepackage{graphicx}
\usepackage{amsmath}
\usepackage{amssymb}
\usepackage{setspace}
\usepackage{natbib}
\usepackage{hyperref}
\usepackage{titlesec}
\usepackage{fancyhdr}
\usepackage{algorithm}
\usepackage{algorithmic}

% Page setup
\onehalfspacing

% Title information
\title{\textbf{Research Proposal:\\
A Data-Driven Distributional Approach to Gait Assessment: Quantifying Deviation from Normal Gait Patterns}}
\author{Stimuloop Research Consortium}
\date{\today}

\begin{document}

\maketitle

\begin{abstract}
    We propose a data-driven gait assessment metric that quantifies patient improvement in neurological rehabilitation by modeling complete gait cycles as probability distributions. The method learns normative patterns from healthy controls and measures patient deviation using distributional distance metrics. We process 4D spatio-temporal motion capture data, segment into gait cycles, and construct individual and population-level distributions. This approach provides objective, comprehensive assessment of gait patterns applicable across rehabilitation scenarios.
\end{abstract}\newpage
\tableofcontents
\newpage

\section{Introduction}

Current gait assessment methods rely on discrete parameters or subjective scales, failing to capture the holistic nature of human locomotion \citep{baker2013gait}. Traditional approaches examine isolated parameters, missing complex multidimensional changes in neurological conditions \citep{whittle2014gait}.

We propose modeling gait as probability distributions over complete cycles to capture natural variability and movement quality \citep{hausdorff2007gait}. This distributional approach addresses the gap in comprehensive gait assessment tools that can quantify deviation from healthy patterns and detect rehabilitation improvements \citep{horst2019explaining}.

\section{Research Objectives}

\begin{enumerate}
    \item Develop a distributional gait assessment metric quantifying deviation from healthy patterns
    \item Establish normative gait distributions with demographic conditioning (age, sex, height)
    \item Validate metric sensitivity for detecting improvements in stroke and Parkinson's patients
    \item Compare distributional distance measures for optimal gait quality quantification
    \item Create a robust assessment tool applicable across rehabilitation scenarios
\end{enumerate}\section{Literature Review}

\subsection{Current Approaches}

Traditional gait analysis uses discrete measurements (stride length, cadence, joint angles) that fail to capture coordinated locomotion dynamics \citep{baker2013gait, whittle2014gait}.

\subsection{Probabilistic Approaches}

Probabilistic models show promise for comprehensive gait representation. \citet{gholami2019probabilistic} demonstrated hidden Markov models for temporal dependencies. \citet{stihi2025hierarchical} used Hierarchical Variational Sparse Heteroscedastic Gaussian Processes for multiple sclerosis analysis, capturing variability patterns and neurological impairment hallmarks.

\subsection{Generative Models}

Recent AI-driven generative models address data scarcity in clinical populations. \citet{peppes2023foggan} developed FoGGAN for Parkinson's episodes. \citet{adeli2025gaitgen} introduced GAITGen for conditional 3D gait generation. \citet{yamada2025synthetic} demonstrated physics-based approaches achieving superior performance through synthetic data diversity. \citet{rezvani2025diffusegaitnet} applied diffusion models for Parkinson's severity assessment.

\subsection{Physics-Constrained Models}

\citet{takeishi2021physics} developed physics-integrated VAEs embedding biomechanical models for realistic gait generation. \citet{ghosh2025gait} validated biomechanical feasibility requirements, ensuring plausible and interpretable synthetic patterns.

\subsection{Distributional Assessment Framework}

Our approach leverages distributional perspectives for assessment rather than generation. We focus on measuring deviation from normative patterns using probabilistic representations \citep{adeli2025gaitgen, takeishi2021physics}. This provides direct quantitative measures of gait abnormality while capturing variability inherent in neurological conditions. \citet{zhang2025gaitmotion} demonstrated 65\% improvement in estimation accuracy through distributional modeling, supporting our hypothesis that comprehensive distributional representations provide more sensitive assessment tools.



\subsection{Distance Measures}

Distance metric selection critically impacts assessment sensitivity and interpretability. We will evaluate information-theoretic measures, geometric distances respecting gait structure, and optimal transport distances. Selection criteria include computational efficiency, clinical sensitivity, noise robustness, and practitioner interpretability.

\subsection{Clinical Context}

Post-stroke gait shows asymmetry and compensatory patterns \citep{dobkin2005rehabilitation}. Parkinson's disease presents reduced step length and increased variability \citep{keus2007european}. Traditional metrics miss complex interactions between movement aspects. Rehabilitation requires sensitive measures detecting gradual improvements \citep{mehrholz2017treadmill}.

\section{Methodology}

\subsection{Framework Overview}

We model gait as probability distributions over complete cycles with three components: (1) normative distribution learning, (2) patient-specific estimation, and (3) distributional distance computation.

\subsection{Data Representation}

Gait data is represented as 4-dimensional spatio-temporal graphs, where each node corresponds to an anatomical landmark with 3D spatial coordinates $(x, y, z)$ and temporal information $t$. For a single gait cycle, we have:

\begin{equation}
    G_i = \{(x^j_i(t), y^j_i(t), z^j_i(t), t) : j \in \mathcal{J}, t \in [0, T_i]\}
\end{equation}

where $i$ indexes the gait cycle, $j$ indexes anatomical landmarks in the set $\mathcal{J}$, and $T_i$ is the duration of cycle $i$.

\subsection{Gait Cycle Segmentation and Normalization}

Raw motion capture recordings undergo automatic segmentation into individual gait cycles using heel-strike detection algorithms. Each cycle is temporally normalized to a standard duration to enable comparison across different walking speeds:

\begin{equation}
    \tilde{G}_i = \{(x^j_i(\tau), y^j_i(\tau), z^j_i(\tau), \tau) : j \in \mathcal{J}, \tau \in [0, 1]\}
\end{equation}

where $\tau = t/T_i$ represents normalized time within the gait cycle.

\subsection{Individual Gait Distribution Estimation}

For each person $p$, let $\mathcal{G}_p = \{\tilde{G}_{p,1}, \tilde{G}_{p,2}, \ldots, \tilde{G}_{p,n_p}\}$ represent normalized gait cycles. We estimate $P_p(\tilde{G})$ using density modeling techniques selected based on dimensionality, sample size, and distributional characteristics. Approaches include parametric, non-parametric, or hybrid methods optimized for variability capture and computational tractability.

\subsection{Normative Distribution Construction}

The normative distribution is constructed from healthy control subjects. We consider two approaches:

\subsubsection{Unconditional Normative Distribution}
\begin{equation}
    P_{norm}(\tilde{G}) = \frac{1}{N} \sum_{p=1}^{N} P_p(\tilde{G})
\end{equation}

where $N$ is the number of healthy control subjects.

\subsubsection{Conditional Normative Distribution}
For demographic conditioning on $\mathbf{z} = (age, sex, height)^T$:

\begin{equation}
    P_{norm}(\tilde{G}|\mathbf{z}) = \sum_{p=1}^{N} w_p(\mathbf{z}) P_p(\tilde{G})
\end{equation}

where weights $w_p(\mathbf{z})$ incorporate demographic covariates using appropriate statistical methods.

\subsection{Deviation Quantification}

We quantify gait deviation through distributional distance measures comparing patient distributions to normative patterns. We evaluate information-theoretic, geometric, optimal transport, and statistical distance measures based on clinical sensitivity, computational efficiency, noise robustness, interpretability, and stability across populations.

\subsection{Study Design and Participants}

\subsubsection{Phase 1: Normative Database Construction}
\begin{itemize}
    \item \textbf{Healthy controls}: $N = 30$ participants
    \item \textbf{Age range}: 40-70 years, stratified by decade
    \item \textbf{Gender}: Balanced representation
    \item \textbf{Exclusion criteria}: No history of neurological, orthopedic, or cardiovascular conditions affecting gait
\end{itemize}

\subsubsection{Phase 2: Clinical Validation}
\begin{itemize}
    \item \textbf{Stroke patients}: $n = 15$, recruited from rehabilitation centers
    \item \textbf{Parkinson's patients}: $n = 15$, recruited from movement disorder clinics
    \item \textbf{Stimuloop participants}: $n = 30$, undergoing specific rehabilitation protocol
    \item \textbf{Assessment timeline}: Baseline, 4 weeks, 8 weeks, 12 weeks post-intervention
\end{itemize}

\subsection{Validation}

We assess construct validity (correlation with clinical measures), sensitivity (healthy vs. patient differences), responsiveness (intervention detection), and test-retest reliability. Additional validation includes synthetic data testing \citep{yamada2025synthetic}, generative augmentation robustness \citep{zhang2025gaitmotion}, and cross-modal consistency with generative model representations.

\section{Expected Outcomes}

\begin{enumerate}
    \item Validated distributional gait assessment metric quantifying deviation from normative patterns
    \item Normative database with demographic conditioning capabilities
    \item Clinical validation demonstrating sensitivity to rehabilitation improvements
    \item Enhanced objective tracking of rehabilitation progress
\end{enumerate}

\subsection{Primary Outcome}

The \textbf{Gait Deviation Index (GDI)}:

\begin{equation}
    GDI_{patient} = \min\{D(P_{patient}, P_{norm}(\cdot|\mathbf{z}_{patient})), D(P_{patient}, P_{norm})\}
\end{equation}

where $D$ is the optimal distance measure, quantifying likelihood of gait patterns under healthy population distribution.


% Bibliography
\bibliographystyle{apalike}
\bibliography{references}

\end{document}
