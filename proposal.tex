\documentclass[12pt,a4paper]{article}

% Packages
\usepackage[utf8]{inputenc}
\usepackage[margin=1in]{geometry}
\usepackage{graphicx}
\usepackage{amsmath}
\usepackage{amssymb}
\usepackage{setspace}
\usepackage{natbib}
\usepackage{hyperref}
\usepackage{titlesec}
\usepackage{fancyhdr}
\usepackage{algorithm}
\usepackage{algorithmic}

% Page setup
\onehalfspacing

% Title information
\title{\textbf{Research Proposal:\\
Development and Validation of the Distributional Gait Assessment Index: A Novel Probabilistic Approach to Quantifying Gait Abnormality}}
\author{Stimuloop Research Consortium}
\date{\today}

\begin{document}

\maketitle

\begin{abstract}
    We propose the \textbf{Distributional Gait Assessment Index (DGAI)}, a novel data-driven metric that quantifies patient improvement in neurological rehabilitation by measuring deviation from normative gait patterns. Unlike the existing Gait Deviation Index which relies on discrete kinematic parameters, the DGAI models complete gait cycles as probability distributions, learns normative patterns from healthy controls, and measures patient deviation using optimal distributional distance metrics. By processing 4D spatio-temporal motion capture data and constructing individual and population-level distributions, the DGAI provides objective, comprehensive assessment of gait quality that captures natural movement variability. This distributional approach offers a sensitive tool for tracking rehabilitation progress across diverse neurological conditions, addressing critical gaps in current gait assessment methodologies.
\end{abstract}\newpage
\tableofcontents
\newpage

\section{Introduction}

Current gait assessment methods rely on discrete parameters or subjective scales, failing to capture the holistic nature of human locomotion \citep{baker2013gait}. Traditional approaches examine isolated parameters, missing complex multidimensional changes in neurological conditions \citep{whittle2014gait}.

We propose the \textbf{Distributional Gait Assessment Index (DGAI)}, which models gait as probability distributions over complete cycles to capture natural variability and movement quality \citep{hausdorff2007gait}. While the existing Gait Deviation Index provides valuable kinematic assessment, the DGAI addresses the critical gap in comprehensive distributional gait assessment tools by quantifying deviation from healthy patterns and providing sensitive detection of rehabilitation improvements \citep{horst2019explaining}. This novel metric represents a paradigm shift from discrete parameter analysis to holistic distributional assessment of human locomotion.

\section{Research Objectives}

\textbf{Primary Objective:}
\begin{itemize}
    \item Develop and validate the \textbf{Distributional Gait Assessment Index (DGAI)} as a comprehensive distributional metric for quantifying deviation from healthy gait patterns
\end{itemize}

\textbf{Secondary Objectives:}
\begin{enumerate}
    \item Establish robust normative gait distributions with demographic conditioning (age, sex, height, weight)
    \item Validate DGAI sensitivity for detecting rehabilitation improvements in stroke and Parkinson's patients
    \item Optimize distributional distance measures for maximal clinical sensitivity and interpretability
    \item Demonstrate DGAI reliability and responsiveness across diverse rehabilitation scenarios
    \item Create a standardized assessment protocol for clinical implementation of the DGAI
\end{enumerate}\section{Literature Review}

\subsection{Current Approaches}

Traditional gait analysis uses discrete measurements (stride length, cadence, joint angles) that fail to capture coordinated locomotion dynamics \citep{baker2013gait, whittle2014gait}.

\subsubsection{Existing Gait Deviation Index}

The Gait Deviation Index (GDI), developed by \citet{schwartz2008gait}, represents a significant advancement in gait assessment by providing a single summary measure of overall gait pathology. The GDI uses principal component analysis of nine key kinematic variables from 3D gait analysis to create a multivariate index, where 100 represents typical gait and lower scores indicate greater deviation. While the GDI has proven valuable for clinical assessment and research \citep{baker2012gait}, it is limited to discrete kinematic parameters and does not capture the full distributional characteristics of gait patterns or the natural variability inherent in human locomotion \citep{rozumalski2010crouch}.

\subsection{Probabilistic Approaches}

Probabilistic models show promise for comprehensive gait representation. \citet{gholami2019probabilistic} demonstrated hidden Markov models for temporal dependencies. \citet{stihi2025hierarchical} used Hierarchical Variational Sparse Heteroscedastic Gaussian Processes for multiple sclerosis analysis, capturing variability patterns and neurological impairment hallmarks.

\subsection{Generative Models}

Recent AI-driven generative models address data scarcity in clinical populations. \citet{peppes2023foggan} developed FoGGAN for Parkinson's episodes. \citet{adeli2025gaitgen} introduced GAITGen for conditional 3D gait generation. \citet{yamada2025synthetic} demonstrated physics-based approaches achieving superior performance through synthetic data diversity. \citet{rezvani2025diffusegaitnet} applied diffusion models for Parkinson's severity assessment.

\subsection{Physics-Constrained Models}

\citet{takeishi2021physics} developed physics-integrated VAEs embedding biomechanical models for realistic gait generation. \citet{ghosh2025gait} validated biomechanical feasibility requirements, ensuring plausible and interpretable synthetic patterns.

\subsection{Distributional Assessment Framework}

Our approach leverages distributional perspectives for assessment rather than generation. We focus on measuring deviation from normative patterns using probabilistic representations \citep{adeli2025gaitgen, takeishi2021physics}. This provides direct quantitative measures of gait abnormality while capturing variability inherent in neurological conditions. \citet{zhang2025gaitmotion} demonstrated 65\% improvement in estimation accuracy through distributional modeling, supporting our hypothesis that comprehensive distributional representations provide more sensitive assessment tools.

\subsubsection{Beyond the Traditional Gait Deviation Index}

While the established GDI provides valuable clinical insights through kinematic parameter analysis, it has inherent limitations. The GDI reduces complex gait patterns to a single score based on nine discrete variables, potentially missing subtle distributional changes and natural movement variability that characterize neurological conditions. Our proposed Distributional Gait Assessment Index (DGAI) addresses these limitations by:

\begin{itemize}
    \item Modeling complete gait cycles as probability distributions rather than discrete parameters
    \item Capturing natural movement variability and temporal dynamics
    \item Providing sensitivity to distributional changes that may precede kinematic abnormalities
    \item Offering complementary information to the existing GDI for comprehensive assessment
\end{itemize}

The DGAI is designed to work alongside, not replace, the established GDI, providing a more comprehensive view of gait quality through distributional analysis.



\subsection{Distance Measures}

Distance metric selection critically impacts assessment sensitivity and interpretability. We will evaluate information-theoretic measures, geometric distances respecting gait structure, and optimal transport distances. Selection criteria include computational efficiency, clinical sensitivity, noise robustness, and practitioner interpretability.

\subsection{Clinical Context}

Post-stroke gait shows asymmetry and compensatory patterns \citep{dobkin2005rehabilitation}. Parkinson's disease presents reduced step length and increased variability \citep{keus2007european}. Traditional metrics miss complex interactions between movement aspects. Rehabilitation requires sensitive measures detecting gradual improvements \citep{mehrholz2017treadmill}.

\section{Methodology}

\subsection{Framework Overview}

We model gait as probability distributions over complete cycles with three components: (1) normative distribution learning, (2) patient-specific estimation, and (3) distributional distance computation.

\subsection{Data Representation}

Gait data is represented as 4-dimensional spatio-temporal graphs, where each node corresponds to an anatomical landmark with 3D spatial coordinates $(x, y, z)$ and temporal information $t$. For a single gait cycle, we have:

\begin{equation}
    G_i = \{(x^j_i(t), y^j_i(t), z^j_i(t), t) : j \in \mathcal{J}, t \in [0, T_i]\}
\end{equation}

where $i$ indexes the gait cycle, $j$ indexes anatomical landmarks in the set $\mathcal{J}$, and $T_i$ is the duration of cycle $i$.

\subsection{Gait Cycle Segmentation and Normalization}

Raw motion capture recordings undergo automatic segmentation into individual gait cycles using heel-strike detection algorithms. Each cycle is temporally normalized to a standard duration to enable comparison across different walking speeds:

\begin{equation}
    \tilde{G}_i = \{(x^j_i(\tau), y^j_i(\tau), z^j_i(\tau), \tau) : j \in \mathcal{J}, \tau \in [0, 1]\}
\end{equation}

where $\tau = t/T_i$ represents normalized time within the gait cycle.

\subsection{Individual Gait Distribution Estimation}

For each person $p$, let $\mathcal{G}_p = \{\tilde{G}_{p,1}, \tilde{G}_{p,2}, \ldots, \tilde{G}_{p,n_p}\}$ represent normalized gait cycles. We estimate $P_p(\tilde{G})$ using density modeling techniques selected based on dimensionality, sample size, and distributional characteristics. Approaches include parametric, non-parametric, or hybrid methods optimized for variability capture and computational tractability.

\subsection{Normative Distribution Construction}

For healthy controls $\mathcal{H} = \{P_1, P_2, \ldots, P_N\}$, we construct the normative distribution $P_{norm}(\tilde{G}|\mathbf{z})$.

\subsubsection{Distribution Aggregation}
We evaluate:
\begin{itemize}
    \item Gaussian mixture models: $P_{norm} = \sum_{k=1}^K \pi_k \mathcal{N}(\mu_k, \Sigma_k)$
    \item Kernel density estimation with adaptive bandwidth
    \item Variational autoencoders for latent space modeling
    \item Weighted ensemble averaging
\end{itemize}

\subsubsection{Demographic Conditioning}
We will also explore conditioning on demographic variables $\mathbf{z} = \{age, sex, height, weight\}$.
Methods include similarity-based weighting schemes and parametric conditioning approaches to ensure appropriate normative comparisons across different population characteristics.

\subsection{The Distributional Gait Assessment Index (DGAI)}

The \textbf{Distributional Gait Assessment Index (DGAI)} is the core innovation of this research, providing a unified metric for quantifying gait abnormality through distributional analysis. Unlike the existing Gait Deviation Index which focuses on discrete kinematic variables, the DGAI captures the comprehensive deviation of a patient's gait pattern from normative healthy patterns while accounting for natural demographic variation and movement variability.

\subsubsection{DGAI Definition}

The DGAI for a patient $p$ is defined as:

\begin{equation}
    DGAI_p = \min\{D(P_p, P_{norm}(\cdot|\mathbf{z}_p)), D(P_p, P_{norm})\}
\end{equation}

where:
\begin{itemize}
    \item $P_p$ is the patient's estimated gait distribution
    \item $P_{norm}(\cdot|\mathbf{z}_p)$ is the demographic-conditioned normative distribution
    \item $P_{norm}$ is the general population normative distribution
    \item $D(\cdot, \cdot)$ is the optimal distributional distance measure
    \item $\mathbf{z}_p$ represents patient demographic characteristics
\end{itemize}

The minimum operation ensures robustness by selecting the most appropriate normative comparison based on demographic similarity and distributional characteristics.

\subsubsection{DGAI Interpretation}

The DGAI provides an intuitive scale where:
\begin{itemize}
    \item \textbf{DGAI $\approx$ 0}: Gait pattern indistinguishable from healthy controls
    \item \textbf{DGAI $>$ 0}: Increasing deviation from normative patterns
    \item \textbf{Higher values}: Greater gait abnormality requiring intervention
\end{itemize}

\subsubsection{Distance Measure Selection}

We systematically evaluate multiple distributional distance measures for optimal DGAI performance:

\textbf{Information-Theoretic Measures:}
\begin{itemize}
    \item Kullback-Leibler divergence: $D_{KL}(P_p \| P_{norm}) = \int P_p(x) \log \frac{P_p(x)}{P_{norm}(x)} dx$
    \item Jensen-Shannon divergence for symmetric comparison
    \item Mutual information for dependency structure
\end{itemize}

\textbf{Geometric Distances:}
\begin{itemize}
    \item Wasserstein distance respecting gait manifold structure
    \item Hellinger distance for probabilistic similarity
    \item Maximum Mean Discrepancy for kernel-based comparison
\end{itemize}

\textbf{Statistical Measures:}
\begin{itemize}
    \item Energy distance for distribution comparison
    \item Cramér-von Mises test statistics
    \item Anderson-Darling measures for tail sensitivity
\end{itemize}

Selection criteria prioritize clinical sensitivity, computational efficiency, noise robustness, practitioner interpretability, and stability across diverse populations.

\subsubsection{DGAI Validation Framework}

We establish comprehensive validation protocols for the DGAI:

\textbf{Construct Validity:} Correlation with established clinical measures (Berg Balance Scale, Fugl-Meyer Assessment, UPDRS-III)

\textbf{Discriminant Validity:} Ability to distinguish healthy controls from patients across neurological conditions

\textbf{Sensitivity:} Detection of subtle changes during rehabilitation progress

\textbf{Specificity:} Minimal false positives in healthy population variation

\textbf{Responsiveness:} Tracking rehabilitation improvements over time

\textbf{Reliability:} Test-retest consistency and inter-session stability

\subsection{Study Design and Participants}

\subsubsection{Phase 1: Normative Database Construction}
\begin{itemize}
    \item \textbf{Healthy controls}: $N = 30$ participants
    \item \textbf{Age range}: 40-70 years, stratified by decade
    \item \textbf{Gender}: Balanced representation
    \item \textbf{Exclusion criteria}: No history of neurological, orthopedic, or cardiovascular conditions affecting gait
\end{itemize}

\subsubsection{Phase 2: Clinical Validation}
\begin{itemize}
    \item \textbf{Stroke patients}: $n = 15$, recruited from rehabilitation centers
    \item \textbf{Parkinson's patients}: $n = 15$, recruited from movement disorder clinics
    \item \textbf{Stimuloop participants}: $n = 30$, undergoing specific rehabilitation protocol
    \item \textbf{Assessment timeline}: Baseline, 4 weeks, 8 weeks, 12 weeks post-intervention
\end{itemize}

\subsection{Validation}

We assess construct validity (correlation with clinical measures), sensitivity (healthy vs. patient differences), responsiveness (intervention detection), and test-retest reliability. Additional validation includes synthetic data testing \citep{yamada2025synthetic}, generative augmentation robustness \citep{zhang2025gaitmotion}, and cross-modal consistency with generative model representations.

\section{Expected Outcomes}

\subsection{Primary Outcome: The Validated Distributional Gait Assessment Index}

The primary deliverable of this research is a fully validated \textbf{Distributional Gait Assessment Index (DGAI)} that provides:

\begin{itemize}
    \item \textbf{Quantitative Assessment}: Objective measurement of gait deviation from normative patterns with established clinical thresholds
    \item \textbf{Clinical Sensitivity}: Demonstrated ability to detect subtle rehabilitation improvements in stroke and Parkinson's patients
    \item \textbf{Demographic Robustness}: Age, sex, and anthropometric conditioning for appropriate normative comparisons
    \item \textbf{Standardized Protocol}: Ready-to-implement assessment methodology for clinical settings
\end{itemize}

\subsection{Supporting Deliverables}

\begin{enumerate}
    \item \textbf{Normative Database}: Comprehensive normative gait distributions with demographic conditioning capabilities, serving as the foundation for DGAI computation
    \item \textbf{Validation Evidence}: Clinical validation demonstrating DGAI sensitivity, specificity, and responsiveness across neurological conditions
    \item \textbf{Implementation Guidelines}: Standardized protocols for DGAI assessment, interpretation, and clinical decision-making
    \item \textbf{Software Framework}: Open-source implementation enabling widespread clinical adoption of the DGAI
\end{enumerate}

\subsection{Clinical Impact}

The DGAI will transform gait assessment by providing:
\begin{itemize}
    \item Enhanced objective tracking of rehabilitation progress
    \item Sensitive detection of subtle therapeutic improvements
    \item Standardized outcome measurement across rehabilitation centers
    \item Evidence-based optimization of intervention protocols
    \item Improved patient engagement through quantifiable progress metrics
\end{itemize}


% Bibliography
\bibliographystyle{apalike}
\bibliography{references}

\end{document}
